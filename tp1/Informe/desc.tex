\section{Descripci\'on de la m\'aquina a utilizar}

El sistema que proponemos, puede subdividirse en dos estad\'ios: uno que es el que se relaciona al centro de c\'omputos; y otro que se corresponde con la emisi\'on de los votos.\\

El primero es un software utilizable en PC's tradicionales, que admite carga del padr\'on; designaci\'on de presidentes de mesa; distribuci\'on de presidentes de mesa y electores en las distintas escuelas y mesas del pa\'is; cargar los candidatos de cada partido pol\'itico; proveer constrase\~nas para los fiscales generales de cada escuela; recibir los resultados de las elecciones de cada escuela y publicar toda la informaci\'on necesaria en una p\'agina web.\\

El segundo consta de una m\'aquina especialmente dise\~nada para actos de sufragio. Cada m\'aquina impresora de voto posee una pantalla t\'actil en la que se muestran las distintas opciones posibles a votar seg\'un el modo en el que este iniciada; una ranura para insertar una boleta; una impresora que grabar\'a la decisi\'on del elector en la boleta y en el chip de la misma; un lector de c\'odigos de boleta para la etapa del recuento de votos y para que el elector verifique la correctitud de su voto; una entrada para auriculares; un espacio para bater\'ia externa con su bater\'ia cargada y colocada (la cual posee una duraci\'on de tres horas); una entrada para conexi\'on telef\'onica a internet. En la etapa de conteo, la m\'aquina lleva cuenta de los votos que se vayan verificando y no sean impugnados
