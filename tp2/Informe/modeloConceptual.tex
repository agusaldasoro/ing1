\subsection{Modelo Conceptual}

\subsubsection{OCL}


\subsubsection*{Mesa}

\textit{context Mesa}
\begin{enumerate}
\item  No hay dos mesas con el mismo fiscal.

$inv: 
self.Fiscal \rightarrow forall(f1 | Mesa.AllInstances() \rightarrow select(m1 | m1 <> self) \rightarrow forall(m2 | m2.Fiscal \rightarrow forall(f2 | f1 <> f2)))$

\item No hay dos mesas con electores en común.

$self.Electores \rightarrow forAll(e_1 | mesa.AllInstances() \rightarrow forAll(m_1 | m_1 <> self$ implies $m_1.Electores  \rightarrow forAll(e_2 | e_1 <> e_2))$

\item No hay dos mesas con el mismo presidente de mesa.

$mesa.AllInstances \rightarrow forAll(m_1, m_2 | m_1<m_2> \Rightarrow m_1.presidente <> m_2.presidente)$

\item Los presidentes de mesa votan en la misma mesa que son presidentes.

$mesa.Presidentes  \rightarrow forAll(p | self.Electores  \rightarrow select(v | p.dni == v.dni).size() == 1)$

\item La boleta con conteo realmente tiene la información de las boletas de la urna (o sea la suma de los candidatos que hayan votado).

$self.BoletaConConteo.Votos == CONCAT(self.Urna.Contiene.Voto \rightarrow forAll(b | b.voto)) $

\item Los id de las mesas son únicos.

$Mesa.AllInstances() \rightarrow select(m_1, m_2 | m_1.numero <> m_2.numero)$

\item \#Boletas en Blanco + \#Boletas con voto grabado $>=$ \#Votantes de la mesa

$self.Lote \rightarrow Size() + self.Urna.Contiene \rightarrow Size () \geq self.Electores \rightarrow Size()$

\item \#Boletas con voto grabado $<=$ \#Votantes de la mesa

$self.Urna.Contiene  \rightarrow  Size() \leq self.Electores \rightarrow Size()$

\end{enumerate}

\subsubsection*{Boleta}
\begin{enumerate}
\item Los id de las boletas son únicos.

$Boleta.AllInstances() \rightarrow select(b_1, b_2 | b_1.troquel <> b_2.troquel)$

\item Si un elector tiene una boleta en blanco, entonces esta boleta está en la mesa que el elector vota.

\item Si un elector tiene una boleta con voto grabado, entonces esta boleta está en la urna que el elector vot\'o.


\end{enumerate}


\subsubsection*{Voto}
\begin{enumerate}
\item No tiene dos candidatos con la misma postulación.
\end{enumerate}

\subsubsection*{Urna}
\begin{enumerate}
\item Las boletas con voto grabado de la urna son sólo de votantes de la mesa que figura la urna.
\item En una misma urna no hay dos boletas con el mismo elector.    
\item Los id de las urnas son únicos.
\end{enumerate}

\subsubsection*{Escuela}
\begin{enumerate}
\item Las mesas de una escuela no están en otra.
\item Los id de las mesas son únicos.
\end{enumerate}

\subsubsection*{M\'aquina Impresora de Voto}
\begin{enumerate}
\item Los candidatos de todas las máquinas son los mismos.
\item Si la máquina está en modo votación y en uso es porque algún elector está votando.
\item Si la máquina está en modo conteo y en uso es porque no hay ningún elector de la mesa votando o esperando para votar.
\item La máquina de la mesa puede estar en modo votación o conteo pero no en envio.
\end{enumerate}

\subsubsection*{Elector}
\begin{enumerate}
\item Los dni de los     electores son únicos.
\item El elector puede no tener ninguna boleta, tener una vacía o tener una con voto.     \item Podrá tener hasta una boleta.
\item No puede haber un elector y una persona que no figura en el padrón con el mismo dni.
\item La boleta que tiene puede estar en blanco o con voto emitido pero no la boleta del conteo de votos.
\item Si el elector no llegó, está esperando o no lo dejaron votar no tiene boleta.
\item Si el elector está votando tiene una boleta en blanco.
\item Si el elector ya votó tiene una boleta con voto grabado.
\end{enumerate}

\subsubsection*{Partido Pol\'itico}
\begin{enumerate}
\item No hay dos partidos políticos con ningún candidato en común.
\end{enumerate}

\subsubsection*{Centro de C\'omputos}
\begin{enumerate}
\item Hay uno solo.
\end{enumerate}

\subsubsection*{C\'odigo}
\begin{enumerate}
\item El hash del código es el mismo del hash del código fuente.   
\end{enumerate}  

\subsubsection*{Ministerio}
\begin{enumerate}
\item Hay un solo ministerio.
\end{enumerate}

\subsubsection*{Equipo T\'ecnico}
\begin{enumerate}
\item Hay uno sólo.
\end{enumerate}

\subsubsection*{Presidente de Mesa}
\begin{enumerate}
\item Si el presidente de mesa del ministerio está presente, no hay presidente ad hoc.
\item Si la mesa tiene un solo presidente de mesa, tiene que ser el de Ministerio.
\item Si hay dos presidentes, el del ministerio debe estar ausente.
\item Si alguien voto, debe haber votado el presidente mesa antes (El Ad hoc si la mesa tiene dos presidentes, o el del ministerio si tiene uno solo).
\end{enumerate}

\subsubsection*{Candidato}
\begin{enumerate}
\item Los id de los candidatos son únicos.
\end{enumerate}
