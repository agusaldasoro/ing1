Lo primero que se puede notar en el diagrama de actividad, es la división en 2 partes de esta sección. Por un lado, tenemos la verificación del código y la carga del padrón. Luego, pasamos a la carga de candidatos y asignaciòn de escuelas. Seguido de la asignaciòn de presidentes de mesa. Fionalmente se prepara todo para el día de la elección. Es importante notar como especificamos anteriormente que el voto del presidente de mesa es parte de esta instancia. 

Todas etapas tienen limites temporales definidos en relación a la elección, para modelarlos utilizaremos máquinas de estado finitas.

Finalmente en este diagramo definimos la acciòn de votar, si bien es algo de la etapa siguiente y hablaremos más de ella posteriormente, al definir el voto del presidente de mesa como el ultimo paso del proceso de preparaciòn fue necesario definirlo aquí.
