\section{Vistas}

Pasaremos a ahora a presentar los distintos diagramas para continuar modelizando nuestro problema.

En una primera instancia presentaremos el diagrama de clases, si bien presentamos el diagrama no estará acompañado de sus predicados de OCL, estos se presentaran posteriormente en la vista de las distintas etapas. 
Como dijimos anteriormente, presentaremos nuestra solución dividiendola en 3 etapas: Preparación, sufragio y conteo. 

Contamos con 5 tipos de diagramas, analizemos un poco las ventajas y desventajas de cada uno antes de comenzar.

\paragraph{Diagrama de contexto} 

El diagrama de contexto es una de las piedras angulares de este trabajo, nos permite observar las interacciones entre agentes. A su vez, no tenemos orden de las acciones ni una relación temporal lo cual nos impide modelizar muchas situaciones.

\paragraph{Diagrama de clases y OCL}

El diagrama de clases nos presenta una visión general de lo que vamos a modelizar, a su vez nos define un poco más las relaciones entre las clases de nuestro sistema. Permite dar una panorama general de como se comportan las clase entre ellas y que se le atribuye a cada una. Al mismo tiempo, el diagrama no permite expresar todo, para esto usaremos OCL, un lenguaje que expande al diagrama y nos permite expresar más relaciones entre clases.

\paragraph{Diagrama de casos de uso}

Este diagrama nos permite describir la interacción entre el sistema y los distintos actores 
externos. Junto a la descripción de los casos de uso, construye un detalle importante para la 
construcción del sistema.

\paragraph{Diagrama de Actividad}

El diagrama de actividad pone un orden relativo en los casos de uso. Y los relaciona aún más con los actores.

\paragraph{Máquinas de estado finitas}

La máquina de estados permite la composición entre varios actores con el mismo comportamiento y a su vez nos permite dar un orden temporal de los eventos. 


Presentaremos primero el diagra de contexto y el diagrama de clases.

\subsection{Diagrama de Contexto}
\todo[inline]{Aca va el diagrama de contexto}


\subsection{Diagrama de Clases}
\todo[inline]{Aca va el diagrama de clases}

Ya presentados el diagrama de contexto y el de clases podemos pasar a analizar para cada una de las etapas presentadas en la introducción mediante los distintos diagramas ya presentados.

\subsection{Preparación}


\subsubsection{Diagrama de actividad}

\begin{figure}[H]
\centering
%\includegraphics[scale=0.45]{}
%\captionof{figure}{Diagrama de actividad}
\end{figure}
\todo[inline]{Aca va el diagrama de actividad (ojo con la escala)}

\subsubsection{Máquina de estado finito}

Para esta sección fu necesario construir 2 máquinas de estado.

\begin{figure}[H]
\centering
%\includegraphics[scale=0.45]{}
%\captionof{figure}{Máquina de fechas limite}
\end{figure}

Esta máquina de estado nos define los tiempos de la etapa, es decir enmarca los tiempos de los casos de uso presentados anteriormente. Es importante notar que las transiciones marcadas en esta FSM corresponden a las acciones mostradas en el diagrama de actividad con el mismo nombre.

\begin{figure}[H]
\centering
%\includegraphics[scale=0.45]{}
%\captionof{figure}{Máquina de elección aleatoria de presidente de mesa}
\end{figure}



\subsubsection{Casos de uso}

Veamos primero el diagrama de casos de uso y luego veremos el detalle de los distintos casos de uso.

Es importante notar que el detalle de casos de uso, nos interioriza mucho en mucho de lo mencionado anteriormente en el diagrama de actividad. 

\todo[inline]{Acá va el diagrama de casos de uso}

Al ver el diagrama, se nota algo interesante. Muchos agentes distintos terminan teniendo los mismos casos de uso que el elector. Si bien anteriormente tuvimos que distinguirlos y fue importante hacerlo, por sus interacciones esternas al sistema, cuando interactuan con el sistema son lo mismo que un elector.

Nos muestra también la importancia del diagrama de contexto para identificar agentes que quizas no interactuen con el sistema pero son escenciales a la construcción del mis.

\textbf{Caso de Uso: Autenticandose}

\textbf{Actores:} Ministerio

\textbf{Pre:} -

\textbf{Post:} El ministerio se encuentra autenticado en el sistema web.
\begin{table}[h!]
	
 \begin{tabular}{|p{7.5cm} | p{7.5cm}|} 
 \hline
 \textbf{Curso normal} & \textbf{Curso Alternativo} \\
 \hline
 %\hline
 1. El ministerio ingresa al sistema web. & \\
 \hline
 
 2. El sistema le pide usuario y contraseña para ingresar. & \\
 \hline 
 3. El ministerio ingresa usuario y contraseña, y elige la opción “Ingresar”. & \\
 \hline 
 4. La información se valida, y se muestra el menú principal. & 
4.1. La información de inicio no es correcta, por lo que se muestra un mensaje de error. Ir al paso 2.
\\
 \hline 
 5. Fin de CU. & \\

 \hline
 \end{tabular}

\end{table}


\textbf{Caso de uso: Cargando padrón}
\textbf{Actor}: Ministerio
\textbf{Pre}: El ministerio se encuentra autenticado en el sistema, el padrón no fue cargado, y todavía no pasó la fecha límite para modificaciones al padrón.
\textbf{Post}: El padrón se encuentra cargado en el sistema y puede ser consultado desde la web.
\begin{table}[h!]
	
 \begin{tabular}{|p{7.5cm} | p{7.5cm}|} 
 \hline
 \textbf{Curso normal} & \textbf{Curso Alternativo} \\
 \hline
1. El ministerio importa el padrón suministrando un archivo con todos los datos de los ciudadanos capacitados para votar.
El archivo importado es de formato .csv, con las siguientes columnas: Nombres, Apellidos, DNI, Dirección, Ciudad, Provincia y Sexo. & \\
\hline

2. El sistema guarda los datos y los publica en la web, para que sean consultados. &
2.1. Si ocurre un error durante la importación, mostrar mensaje de error e ir al fin del caso de uso. \\
\hline
3. Fin de CU. & \\

 \end{tabular}

\end{table}





\textbf{Caso de Uso: Ingresando nuevo votante}

\textbf{Actores:} Ministerio

\textbf{Pre:} El ministerio se encuentra autenticado en el sistema, el padrón se encuentra cargado en el sistema, y puede ser consultado desde la web.

\textbf{Post:} El padrón se actualiza en el sistema, agregando al nuevo votante.
\begin{table}[h!]
	
 \begin{tabular}{|p{7.5cm} | p{7.5cm}|} 
 \hline
 \textbf{Curso normal} & \textbf{Curso Alternativo} \\
 \hline

1. El ministerio ingresa los datos del nuevo votante en un formulario. & \\
\hline

2. El ministerio elige la opción de “Cargar votantes”. & \\
\hline

3. Se actualizan los datos en el sistema. & 3.1. Si existe un error en la carga, se muestra un mensaje con el error. Ir al paso 2. \\
\hline
4. Fin de CU. & \\
 \hline
 \end{tabular}

\end{table}

\textbf{Caso de uso: Cargando información sobre escuelas}

\textbf{Actor:} Ministerio

\textbf{Pre:} El ministerio se encuentra autenticado en el sistema, y pasó la fecha establecida para la modificación del padrón.

\textbf{Post:} El sistema posee la información de los electores y sus mesas correspondientes, y además designó un presidente de mesa para cada una. Esta información se encuentra consultable en la web.


\begin{table}[h!]
	
 \begin{tabular}{|p{7.5cm} | p{7.5cm}|} 
 \hline
 \textbf{Curso normal} & \textbf{Curso Alternativo} \\
 \hline

1. El ministerio importa la información sobre las escuelas suministrando un archivo con todos los datos de las escuelas. El archivo importado es de formato .csv, con las siguientes columnas: Nombre, Dirección, Ciudad, Provincia y Cantidad de aulas. & \\
\hline

2. El sistema guarda los datos importados y asigna a los electores a las escuelas más cercanas, asignando una mesa a cada uno. &
2.1. Si ocurre un error durante la importación, mostrar mensaje de error e ir al fin del caso de uso. \\
\hline
3. El sistema aleatoriamente elige un presidente de mesa, priorizando a aquellos que no lo fueron previamente. & \\
\hline

4. El sistema ofrece la posibilidad de consultar los presidentes de mesa seleccionados. &
4.1. En caso de ocurrir un error, mostrar mensaje de error e ir a fin del caso de uso. \\
\hline
5. Si se elige la opción, se extiende con el CU Consultando presidentes de mesa. & \\
\hline

6. El sistema publica la información en la web para que pueda ser consultada. & \\
\hline

7. Fin de CU. & \\
\hline

 \end{tabular}

\end{table}






\textbf{Caso de Uso: Consultando presidentes de mesa}

\textbf{Actores:} Ministerio 

\textbf{Pre:} El ministerio se encuentra autenticado en el sistema, las mesas y los presidentes de mesa se encuentra publicados.

\textbf{Post:} El ministerio conoce a los presidentes de mesa.
\begin{table}[h!]
	
 \begin{tabular}{|p{7.5cm} | p{7.5cm}|} 
 \hline
 \textbf{Curso normal} & \textbf{Curso Alternativo} \\
 \hline

1. El ministerio selecciona la pestaña de consulta de presidentes de mesa. & \\
\hline
2. El ministerio consulta en el sistema los presidentes asignados a cada mesa. & \\
\hline
3. Fin de CU. & \\
\hline
 \end{tabular}

\end{table}

\textbf{Caso de Uso: Asignando nuevo presidente de mesa}

\textbf{Actores:} Ministerio 

\textbf{Pre:} El ministerio se encuentra autenticado en el sistema, las mesas y los presidentes de mesa se encuentran publicados, y un presidente de mesa envió una notificación explicando que no puede estar presente. La notificación se envió llamando a la línea gratuita del ministerio.

\textbf{Post:} Se asigna un nuevo presidente a una mesa.
\begin{table}[h!]
	
 \begin{tabular}{|p{7.5cm} | p{7.5cm}|} 
 \hline
 \textbf{Curso normal} & \textbf{Curso Alternativo} \\
 \hline

1. El ministerio activa la opción de una nueva asignación de presidente de mesa para la mesa afectada. & \\
\hline

2.  El sistema aleatoriamente elige un presidente de mesa, descartando al presidente ya elegido, y priorizando a aquellos que no lo fueron previamente. & \\
\hline


3. Fin de CU. & \\
\hline



 \end{tabular}

\end{table}


\textbf{Caso de Uso: Consultando notificaciones de error en código fuente}

\textbf{Actores:} Ministerio 

\textbf{Pre:} El ministerio se encuentra autenticado en el sistema y el código fuente se encuentra publicado.
\textbf{Post:} El ministerio conoce las notificaciones de error sobre el código fuente.


\begin{table}[h!]
	
 \begin{tabular}{|p{7.5cm} | p{7.5cm}|} 
 \hline
 \textbf{Curso normal} & \textbf{Curso Alternativo} \\
 \hline


1. El ministerio selecciona la pestaña de notificaciones. & \\
\hline


2. El ministerio ministerio revisa todas las notificaciones cargadas en el sistema. & \\
\hline


3. Si se decide corregir uno de los errores cargados, se extiende con el CU \textbf{Corrigiendo código fuente}. & \\
\hline


4. Fin de CU. & \\
\hline




 \end{tabular}

\end{table}



\textbf{Caso de Uso: Iniciando votación}

\textbf{Actores:} Presidente de mesa

\textbf{Pre:} Es el día de la votación y ya se encuentra todo preparado en la mesa.

\textbf{Post:} La máquina ya se puede utilizar para votar, y el presidente de mesa ya votó.

\begin{table}[h!]
	
 \begin{tabular}{|p{7.5cm} | p{7.5cm}|} 
 \hline
 \textbf{Curso normal} & \textbf{Curso Alternativo} \\
 \hline

1. El presidente de mesa conecta la máquina de sufragio a la red eléctrica. & \\
\hline
2. El presidente de mesa prende la máquina de sufragio y activa el modo votación. & \\
\hline
3. El presidente de mesa vota de acuerdo a lo explicado en el CU Votando del Elector. & \\
\hline
4. Fin de CU.& \\
\hline



 \end{tabular}

\end{table}



\textbf{Caso de Uso:  Consultando padrón}

\textbf{Actores:} Elector 

\textbf{Pre:} El padrón se encuentra publicado en la web. 

\textbf{Post:}  El elector conoce si se encuentra en el padrón.

\begin{table}[h!]
	
 \begin{tabular}{|p{7.5cm} | p{7.5cm}|} 
 \hline
 \textbf{Curso normal} & \textbf{Curso Alternativo} \\
 \hline

1. El elector ingresa al sistema web público del sistema. & \\
 \hline



2. Se busca en el padrón por su DNI. & \\
 \hline



3. Si se encuentra, ir a 5. & \\
 \hline



4. Si el elector no se encuentra en el padrón, se lo comunica al ministerio, y se extiende con CU Ingresando nuevo votante. & \\
 \hline



5. Fin de CU. & \\
 \hline


 \end{tabular}

\end{table}




\textbf{Caso de Uso: Consultando candidatos}

\textbf{Actores:} Elector 

\textbf{Pre:} Los candidatos fueron publicados.

\textbf{Post:}  El elector conoce los candidatos.
\begin{table}[h!]
	
 \begin{tabular}{|p{7.5cm} | p{7.5cm}|} 
 \hline
 \textbf{Curso normal} & \textbf{Curso Alternativo} \\
 \hline
 
1. El elector ingresa al sistema web público del sistema. & \\
 \hline



2. El elector visita la sección de candidatos de la web. & \\
 \hline


3. Fin de CU. & \\
 \hline


 \end{tabular}

\end{table}


\textbf{Caso de Uso: Consultando código fuente}

\textbf{Actores:} Elector

\textbf{Pre:} El código fuente se encuentra publicado en la web.

\textbf{Post:} El elector ahora conoce el código fuente.

\begin{table}[h!]
	
 \begin{tabular}{|p{7.5cm} | p{7.5cm}|} 
 \hline
 \textbf{Curso normal} & \textbf{Curso Alternativo} \\
 \hline
1. El elector ingresa al sistema web público del sistema. & \\
 \hline


2. El elector navega a la sección del código fuente. & \\
 \hline


3. El elector revisa el código fuente, en busca de posibles errores. & \\
 \hline


4. Si no encuentra errores, ir a 6. & \\
 \hline


5. Si encuentra un error, se extiende con el CU Notificando error en el código fuente. & \\
 \hline


6. Fin de CU. & \\
 \hline

 \end{tabular}

\end{table}



Caso de uso: 
Actor: Elector
Pre: El código fuente se encuentra publicado en la web, el elector se encuentra en el sistema web y ya consultó el código previamente.
Post: En el sistema ahora existe una nueva denuncia del código fuente.
Curso normal
Curso alternativo
1. El elector navega a la sección de denuncias del código fuente. 


2. El elector llena un formulario, explicando las causas de la denuncia. Luego, elige la opción de “Enviar denuncia”.


3. Se valida la información, y se carga en el sistema.
3.1. Si existe un error en la carga del formulario, se muestra un mensaje de error. Ir al fin del caso de uso.
4. Fin de CU.


\textbf{Caso de Uso: Notificando error en código fuente}

\textbf{Actores:}  Elector

\textbf{Pre:} El código fuente se encuentra publicado en la web, el elector se encuentra en el sistema web y ya consultó el código previamente.

\textbf{Post:} En el sistema ahora existe una nueva denuncia del código fuente.
\begin{table}[h!]
	
 \begin{tabular}{|p{7.5cm} | p{7.5cm}|} 
 \hline
 \textbf{Curso normal} & \textbf{Curso Alternativo} \\
 \hline
1. El elector navega a la sección de denuncias del código fuente. & \\
\hline


2. El elector llena un formulario, explicando las causas de la denuncia. Luego, elige la opción de “Enviar denuncia”. & \\
\hline


3. Se valida la información, y se carga en el sistema. & 3.1. Si existe un error en la carga del formulario, se muestra un mensaje de error. Ir al fin del caso de uso. \\
\hline
4. Fin de CU.& \\
\hline
 \end{tabular}

\end{table}






\subsubsection{OCL}





\subsection{Sufragio}

\subsubsection{Máquina de estado finito}

\subsubsection{Diagrama de actividad}


\begin{figure}[H]
\centering
%\includegraphics[scale=0.45]{}
%\captionof{figure}{Diagrama de actividad}
\end{figure}
\todo[inline]{Aca va el diagrama de actividad (ojo ocn la escala)}


\subsubsection{Casos de uso}


\textbf{Caso de Uso: }

\textbf{Actores:} 

\textbf{Pre:} 

\textbf{Post:}
\begin{table}[h!]
	
 \begin{tabular}{|p{7.5cm} | p{7.5cm}|} 
 \hline
 \textbf{Curso normal} & \textbf{Curso Alternativo} \\
 \hline

 \end{tabular}

\end{table}


\subsubsection{OCL}

\subsection{Conteo}

\subsubsection{Máquina de estado finito}

\subsubsection{Diagrama de actividad}

\begin{figure}[H]
\centering
%\includegraphics[scale=0.45]{}
%\captionof{figure}{Diagrama de actividad}
\end{figure}
\todo[inline]{Aca va el diagrama de actividad (ojo ocn la escala)}

\subsubsection{Casos de uso}


\textbf{Caso de Uso: }

\textbf{Actores:} 

\textbf{Pre:} 

\textbf{Post:}
\begin{table}[h!]
	
 \begin{tabular}{|p{7.5cm} | p{7.5cm}|} 
 \hline
 \textbf{Curso normal} & \textbf{Curso Alternativo} \\
 \hline

 \end{tabular}

\end{table}


\subsubsection{OCL}