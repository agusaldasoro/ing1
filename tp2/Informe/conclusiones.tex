\section{Conclusiones}
- Conclusiones: mencionar brevemente de qué formas les resultó más sencillo encarar el TP. Por ejemplo en qué orden realizaron los diagramas, qué aspectos presentaron las mayores dificultades, etc.

Lo primero que se puede decir es que la división del problema en tres partes fue la piedra angular de lograr un trabajo conciso y comprensible, de no haber sido así los diagramas hubieran sido no solo más dificiles de leer sino dificiles de concebir.

A su vez esta división y la construcción de un texto integral describiendo en español el problema nos permitio una construcción en paralelo de todos los diagramas, etapa por etapa. 

Se puede decir que la mayor dificultad fue construir un conjunto de diagramas que se acoplen entre sí y a su vez describan bien lo que queriamos. Lograr esto implico una critica constante de todos los diagramas y la busqueda de soluciones a problemas que fueron apareciendo, principalmente problemas que no habían pensado en primera instancia como puede ser alguna cota temporal necesaria, que aparecieron durante la cosntrucción del diagrama de actividad y que implicaron un cambio en las máquinas de estado finitas.

Concluyendo, presentamos un conjunto de modelos de manera a permitir la siguiente etapa del proyecto	